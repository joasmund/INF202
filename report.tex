\documentclass{article}
\usepackage{amsmath} % For avanserte matematiske funksjoner
\usepackage{graphicx} % For bilder
\usepackage{tocloft} % Forbedret innholdsfortegnelse

\begin{document}

% Title Page
\begin{titlepage}
\centering
\vspace*{1cm} 
{\Large \textbf{Simulation of an Oil Spill in Bay City} \par} 
\vspace{1.5cm} 
{\large
\textit{Jørgen Asmundvaag \\ 
Ivar Eftedal \\ 
Sebastian Sverkmo} \par}
\vspace{1.5cm} 
{\large \textbf{January 2025} \par}
\vspace*{\fill} 
\end{titlepage}
% End title Page

\newpage
\tableofcontents
\newpage

\section{Overall Problem}
The goal is to develop a simulation to model the spread of an oil spill in Bay City, a fictional coastal town. This simulation will predict the oil's movement and its impact on nearby fishing grounds, helping the community mitigate ecological and economic risks.\\
The simulation models the oil's evolution based on an initial distribution and a flow field representing ocean currents. The fishing grounds, located in $[0.0, 0.45] \times [0.0, 0.2]$, are the primary area of concern, with the aim of evaluating oil concentration over time.\\
The initial oil distribution of the problem is:
\begin{equation}
u(t=0, \vec{x}) = \exp\left(-\frac{\|\vec{x} - \vec{x}_\star\|^2}{0.01}\right)
\end{equation}
Here, $u(t=0, \vec{x})$ represents the initial concentration of oil at point $\vec{x}$.

The flow field guiding the oil's movement is:
\begin{equation}
v(\vec{x}) = 
\begin{pmatrix}
y - 0.2x \\
-x
\end{pmatrix}
\end{equation}

\section{Simulation}
\subsection{Mathematical Model}
Explain the equations...
\subsection{Numerical Implementation}
Explain the computational steps...

\section{User Guide}
\subsection{Installation}
- Download the simulation files from the repository.
- Install Python and required packages.
\subsection{Running the Simulation}
- Run `python simulate.py` to execute the simulation.
- Follow on-screen prompts.

\section{Structure}
Describe the structure of the project files...

\section{Development}
Highlight key development steps...

\section{Results}
\begin{figure}[h!]
\centering
\includegraphics[width=0.8\textwidth]{results_graph.png}
\caption{Oil concentration over time in fishing grounds.}
\label{fig:results}
\end{figure}

\end{document}